\documentclass{beamer} % "Beamer" is a word used in Germany to mean video projector. 

\usetheme{Berkeley} % Search online for beamer themes to find your favorite or use the Berkeley theme as in this file.

\usepackage{color} % It may be necessary to set PCTeX or whatever program you are using to output a .pdf instead of a .dvi file in order to see color on your screen.
\usepackage{graphicx} % This package is needed if you wish to include external image files.

\theoremstyle{definition} % See Lesson Three of the LaTeX Manual for more on this kind of "proclamation."
\newtheorem*{dfn}{A Reasonable Definition}               

\title[ER\&RE]{Entity Recognition and Relation Extraction}

\author[Sai Kiran Naragam]{Sai Kiran Naragam\\ N100638}
%\institute{Rajiv Gandhi University of Knowledge Technologies\\ Nuzvid}
\date[\today]{Under the Guidance of\\\textbf{Mr. Uday Kumar Ambati\\Asst. Professor}\\ \textbf{Dept. of Comp. Sci. \& Engg.\\RGUKT Nuzvid}}
%\date{January 6, 2012} 
% Remove the % from the previous line and change the date if you want a particular date to be displayed; otherwise, today's date is displayed by default.

\AtBeginSection[]  % The commands within the following {} will be executed at the start of each section.
{
\begin{frame} % Within each "frame" there will be one or more "slides."  
\frametitle{Presentation Outline} % This is the title of the outline.
\tableofcontents[currentsection]  % This will display the table of contents and highlight the current section.
\end{frame}
} % Do not include the preceding set of commands if you prefer not to have a recurring outline displayed during your presentation.

\begin{document}

\begin{frame} 
\titlepage
\end{frame}

\section{Introduction} % Since this is the start of a new section, our recurring outline will appear here.

\begin{frame} 
\frametitle{Aim}
To recognise entities(Named Entites) in the given text and extracting relationships.
\includegraphics[width=7cm, height=5cm]{ie.png}
\textit{www.nltk.org/book/}
\end{frame}

\section[ER]{Entity Recognition}
\begin{frame}
\frametitle{Intro}
\includegraphics[width=10cm, height=5cm]{ner.png}
\\
Image: \textit{http://www.europeana-newspapers.eu/named-entity-recognition-for-digitised-newspapers/}
\end{frame}

\begin{frame}
\frametitle{Entity Recognition}
We can achieve this in two ways:
\begin{itemize}
\pause \item \texttt{Procedure 1} % Each \pause creates a new slide within the frame.
\begin{itemize}
\item PoS Taggins
\item Named Entity Chunking
\end{itemize}
\pause \item \texttt{Procedure 2}
\begin{itemize}
\item Named Entity Extraction as Tagging
\end{itemize}
\end{itemize}
\end{frame}

\subsection[Tagging]{Parts of Speech Tagging}
\begin{frame}
\frametitle{Intro. to PoS Tagging}
\includegraphics[width=10cm, height=5cm]{postag.png}
\end{frame}
\begin{frame}
\frametitle{PoS Tagging Contd..}
Here we used NLTK's recommended PoS Tagger.It uses the Penn Treebank tagset.
\end{frame}

\subsection[Chunking]{Named Entity Chunking}
\begin{frame}
\frametitle{Intro to Named Entity Chunking}
\includegraphics[width=9cm, height=3cm]{chunk1.png}
\\
\textit{www.nltk.org/book/}
\end{frame}
\begin{frame}
\frametitle{Named Entity Chunking Contd...}
\begin{itemize}
\item We can use \textit{hand-written rules(\textbf{Ex: regular expressions}} to chunk tags in to a named-entity. Or we use a 
trained chunker.
\item In the case we used NLTK's recommended Chunker. Which is trained on ACE( Automatic
 Context Extraction).
 \end{itemize}
\end{frame}

\subsection[ER as Tagging]{Entity Extraction itself as Tagging}
\begin{frame}
\frametitle{Entity Extraction itself as Tagging}
Example: Bikel et. al 1999 (Named Entity Recognition)
\includegraphics[width=9cm, height=5cm]{nerastagging.png}\\
\end{frame}
\begin{frame}
\frametitle{Entity Extraction as Tagging Contd..}
Bikel et al. used Hidden Markov Models to tag.
\includegraphics[width=9cm, height=5cm]{nerastagging2.png}\\
\textit{http://curtis.ml.cmu.edu/w/courses/index.php/}
\end{frame}

\section[RE]{Relation Extraction}
\subsection{Introduction}
\begin{frame}
\frametitle{Intro to Relation Extraction}
\textit{What kind of relationship is there between entities}
\includegraphics[width=7cm, height=3cm]{rel.png}
\end{frame}

\begin{frame}
\begin{itemize}
\frametitle{Intro to Relation Extraction}
\item Here we are using \textit{hand-written patterns(\textbf{regular expressions})} to extract the relationship.
\item We may use the so-called \textit{learning technique}. But we may need huge hand-labeled
 training data(in case of supervised) or very advanced techniques(in case of unsupervised).
\end{itemize}
\end{frame}

\begin{frame}
\frametitle{How ?}
\begin{itemize}
\item Form relation-tuples of the form \textit{(subj, filler, obj)}
\item \textit{subj,obj} are Entities and \textit{filler} is text.
\item Apply rules on \textit{filler} to extract relation.
\end{itemize}
\end{frame}

\begin{frame}
\frametitle{Applications of Relation Extraction}
Concentrated application is
\textbf{Question Answering}\\
\textit{where are different facilities located}\\
\textit{who is employed by what company}.\\
\end{frame}

\section[Summary]{Summary of Work}
\begin{frame}
\frametitle{Summary of Work}
\begin{itemize}
\item Technologies: Python,NLTK.\\
\item Tested: BBC business text.\\
\item Worked only with few class PERSON,LOCATION,ORGANIZATION,GPE.\\
\item Looking forward to work with 7 classes.
\item Started workin with \textbf{Stanford Named Entity Recognizer}.
\item Theoritical study of HMMTagger,Maximum Entropy Markov Model Tagger(\textit{not completed}) and Perceptron Tagger(\textit{not completed}).
\item Generative vs. discriminative learning models.
\end{itemize}
\end{frame}

\section[Future Work]{Future Implementations}
\begin{frame}
\frametitle{Future Implementations}
\begin{itemize}
\item To do everything done so far for Indian languages.
\item Tried but left the work due to some difficulties observed.
\end{itemize}
\end{frame}

\section[]{}
\begin{frame}
\textbf{Thank You}
\end{frame}
\end{document}
