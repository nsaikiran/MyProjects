	
\documentclass[12pt]{report}

\usepackage{amsmath}
\usepackage[colorlinks,linkcolor=black]{hyperref}
\usepackage[margin=2cm]{geometry}
%for images
\usepackage{graphicx}
%\usepackage{wrapfig}
\usepackage{url}

\newcommand*{\Signature}[1]{
	\vspace{3cm}
    \par\noindent\makebox[5cm]{\hrulefill}
    \par\noindent\makebox[5cm][l]{#1}
}


\newcommand*{\plogo}{\fbox{$\mathcal{PL}$}} % Generic publisher logo

%----------------------------------------------------------------------------------------
%	TITLE PAGE
%----------------------------------------------------------------------------------------

\newcommand*{\titleGP}{\begingroup % Create the command for including the title page in the document
\centering % Center all text
\vspace*{\baselineskip} % White space at the top of the page


\rule{\textwidth}{1.6pt}\vspace*{-\baselineskip}\vspace*{2pt} % Thick horizontal line
\rule{\textwidth}{0.4pt}\\[\baselineskip] % Thin horizontal line

{\LARGE \textbf{Named Entity Recognition\\ and \\[0.3\baselineskip] Relation Extraction}}\\[0.2\baselineskip] % Title

\rule{\textwidth}{0.4pt}\vspace*{-\baselineskip}\vspace{3.2pt} % Thin horizontal line
\rule{\textwidth}{1.6pt}\\[\baselineskip] % Thick horizontal line

%\scshape % Small caps
{\Large Bachelor's Thesis\par} % Tagline(s) or further description
 submitted in partial fullfilment of\\
 the requirements for the degree of \\
Bachelor of Technology\\
in\\
\textbf {Computer Science and Engineering}\\
[\baselineskip]
By\\
{\Large SAI KIRAN NARAGAM\\N100638\par}
\\[\baselineskip] % Tagline(s) or further description

{Under the Guidance of\par}
{\textbf {Mr. A.Udaya Kumar\\Asst. Professor\\Dept. of Computer Science and Engineering\\  RGUKT Nuzvid}\par}
\vspace*{2\baselineskip} % Whitespace between location/year and editors


%Edited by \\[\baselineskip]
%{\Large JOHN SMITH \\ JANE SMITH \\ JAMES SMITH\par} % Editor list
%{\itshape The University of California \\ Berkeley\par} % Editor affiliation

\vfill % Whitespace between editor names and publisher logo
\includegraphics[scale=.4]{rguktlogo.jpg}\\
%\plogo \\[0.3\baselineskip] % Publisher logo
{\scshape \textbf{Rajiv Gandhi University of Knowledge Technologies, Nuzvid.\\Krishna District,Andhra Pradesh.}} %\\[0.3\baselineskip] % Year published
%{\large THE PUBLISHER}\par % Publisher

\endgroup}

%\title{Scilab Optimization Toolbox Using Symphony\\Submitted in Fulfilment of\\Summer Internship 2015 Project}
%\author{Keyur Joshi \and Sai Kiran Naragam \and Iswarya Hariharan}
%\date{May-July 2015}

\usepackage{fancyhdr, graphicx}
\renewcommand{\headrulewidth}{0pt}
\fancyhead[R]{\textbf{Rajiv Gandhi University of Knowledge Technologies\\
	(A.P. Government Act 18 of 2008)\\
	RGUKT NUZVID, Krishna District - 521202}\\}
\fancyhead[L]{
\includegraphics[width=1.5cm]{rguktlogo.jpg}
}

\begin{document}

\pagestyle{empty} % Removes page numbers

\titleGP % This command includes the title page

% title page
%\maketitle
\cleardoublepage
\pagenumbering{roman}
\setcounter{page}{1} %reset the page counter
%acknowledgements

%\chapter*{Project Approval}
\newgeometry{left=3cm,right=3cm,top=5cm}


\section*{}
%\begin{center}\textbf{Rajiv Gandhi University of Knowledge Technologies\\(A.P. Government Act 18 of 2008)\\RGUKT-NUZVID, Krishna Dist. - 521202\\}\end{center}

\thispagestyle{fancy}
\begin{center}\makebox[15cm]{\hrulefill}\end{center}
\vspace{1cm}
\begin{center}
\subsection*{CERTIFICATE OF COMPLETION\\\makebox[10cm]{\hrulefill}}

\end{center}
\par This is to certify that the work entitled, \textbf{ Entity  Recognition and Relation Extraction} 
is the bona fied work of \textbf{NARAGAM SAI KIRAN}, ID No: \textbf{N100638}, carried out under my guidance and supervision,
for the partial fulfillment of the requirements for the award of the degree of Bachelor of Technology in Computer Science and Engineering. 

\Signature{Mr. Ambati Udaya Kumar,}\\Assistant Professor,\\Dept.of CSE.

\pagebreak
\newgeometry{left=3cm,right=3cm,top=5cm}
\section*{}
\thispagestyle{fancy}
\begin{center}\makebox[15cm]{\hrulefill}\end{center}
\vspace{1cm}
\begin{center}
\subsection*{CERTIFICATE OF EXAMINATION\\\makebox[10cm]{\hrulefill}}

\end{center}
\par This is to certify that the work entitled, \textbf{"Named Entity Recognition and Relation Extraction"}
is the bonafide work of \textbf{NARAGAM SAI KIRAN}, ID No: \textbf{N100638} and here by accord our approval
 of it as a study carried out and presented in a manner required for its
  acceptance in the partial fulfillment of the requirement for 
  the award of the degree of Bachelor of Technology for which it has been submitted.
   This approval does not necessarily endorse or accept every statement made, opinion expressed or conclusion drawn, as a recorded in this thesis.
    It only signifies the acceptance of this thesis for the purpose for which it has been submitted.

\Signature{Mr. Ambati Udaya Kumar,}\\Assistant Professor,\\Dept.of CSE.
\Signature{Mr. Amit Patel,}\\Lecturer,\\Dept.of CSE.
\pagebreak

\newgeometry{left=3cm,right=3cm,top=5cm}
\section*{}
\thispagestyle{fancy}
\begin{center}\makebox[15cm]{\hrulefill}\end{center}
\vspace{1cm}
\begin{center}
\subsection*{DECLARATION\\\makebox[5cm]{\hrulefill}}

\end{center}
\par I \textbf{NARAGAM SAI KIRAN}, with ID No:\textbf{N100638} hereby declare that t
he project report entitle \textbf{Named Entity Recognition and Relation Extraction} done by me
 under the guidance of \textbf{Mr. Ambati Udaya Kumar,M.Tech} is submitted for the partial
  fulfillment of the requirements for the award of the degree of Bachelor of Technology
   in Computer Science and Engineering during the academic session August 2015 – April 2016 at RGUKT – Nuzvid.

\par I also declare that this project is a result of my own effort and has not been copied or imitated from any source.
 Citations from any websites are mentioned in the references.
\par The results embodied in this project report have not been submitted to any other university or institute for the award of any degree or diploma.

\Signature{N. SAI KIRAN,}\\N100638.\\
\begin{flushright}
Date : \makebox[3cm]{\hrulefill}\\
Palce: \makebox[3cm]{\hrulefill}\\
\end{flushright}
\pagebreak

\newgeometry{margin=2cm}
\chapter*{Acknowledgements}
\par I would like to thank my parents,guide and friends
\chapter*{Abstract}

\par Abstract goes here.

%table of contents
\pagestyle{empty} %get rid of header/footer for toc page
\tableofcontents %put toc in
\cleardoublepage %start new page
\pagestyle{plain} % put headers/footers back on
\pagenumbering{arabic}
\setcounter{page}{1} %reset the page counter

\chapter{Introduction}
When we have large amount of (previous)data we might want to extract some useful information
 out of it, and use it as summary. or we can predict the future events by learning from the
 data at hand.\\
Most of the time data that is available for use in un-structured form like Natural Language Text
rather than structured form like tables. It is easy to extract required information or 
answer a question if the data we are working on has structured form.

But it is difficult to handle unstructured data. Because Natural Language Processing(NLP)
that works on unstructured data is still developing.

The amount of natural language text that is available in electronic 
form is truly staggering, and is increasing every day. 
However, 
the complexity of natural language can make it very difficult to access the information in that text\cite{BookIE}.\\

\par If we instead focus our efforts on a limited set of questions or 
"entity relations," such as "where are different facilities located,
" or "who is employed by what company," we can make significant progress.\cite{BookIE}

\par Here we are trying to understand the given text and find the limited relevant parts of it.
 This is what the researchers called as \textbf{Information Extraction}.
\section{Aim}
\par Identifying named entities and working out the relationship between them using hand-written
 rules with regular expressions. For most of the questions often the answers be named entities.
\begin{figure}[htp]
\centering
\includegraphics[height=2in,width=3in]{ie.png}
\caption{Simple Pipeline Architecture for an Information Extraction System\cite{BookIE}}
\label{IE}
\end{figure}

\chapter{Named Entity Recognition}
\section{Introduction}
\par \textbf{Named Entity Recognition} is an important sub task of \textit{Information Extraction} 
,in this we are going to find and classify (into different classes like PERSON,ORGANIZATION and LOCATION etc.).
 concrete names of people,organizations,locations and quantities etc.
\par We are interested in \textit{Named} Entity Recognition. Because not all entities are 
attached with a name (specific).
\par For the literature survey on named entity recognition, please refer\cite{Rahul}.
\section{Named Entity Recognition as Tagging}
\par \textit{Bikel et. al}\footnote{\url{http://ilk.uvt.nl/~toine/research/bikel-1999.pdf}} mapped the  Named the Entity Recognition problem very directly into tagging problem. Where
where they considered all the named entity classes and an extra NOT-A-NAME tag. 

\begin{figure}[htp]
\centering
\includegraphics[height=3in,width=5in]{bikel.png}
\caption{Stage Diagram of NER as Tagging by Bikel et. al}
\label{NERasT}
\end{figure}

They used hand-tagged corpus to train their model(Hidden Markov Model\cite{hmm}) and some
 word-features to deal with low-frequency words. 
 
\section{Named Entity Recognition with PoS tagging \& Chunking}
\par Now we have lot of parts of speech tagged corpora (especially for english) as we can
 use it for machine translataion and many other applications. Here we are going to use 
 PoS tagging for NER. \par After having natural language senetences with their underlying \textit{
 tag sequences} we group the tags into named entities.

\subsection{PoS Tagging\footnote{For more details:\\Tagging Problems, and Hidden Markov Models of \cite{mc} and POS Tagging of \cite{SNLP}}}
\par Parts of speech tagging problem is to determine the parts of speech of a particular
instance of word. The intuition of PoS tagging is presented in below image \ref{pos}\footnote{Slide from
 The Tagging Problem of \cite{mc}}.
\par Tags may vary depending on corpus we are dealing with. For example, tag set of The Brown Corpus\footnote{\url{https://www.comp.leeds.ac.uk/ccalas/tagsets/brown.html}} and P.O.S tag set of The Penn Treebank\footnote{\url{https://www.ling.upenn.edu/courses/Fall_2003/ling001/penn_treebank_pos.html}}
To check in NLTK, execute and \textit{nltk.help.brown\_tagset()},\textit{nltk.help.upenn\_tagset()} respectively for the brown corpus tagset and Penn Treebank tagset.
\par If we want to write a tagger then we need \textit{large amount of labled corpus}\footnote{For more insights of 
Machine Learning techniques I feel, \cite{Videos} is a good source}. Which in this case
 are Penn Tree Bank tagged corpus or Brown corpus.
\par For more insights on writing a parts of speech tagger in NLTK, please refer\cite{tag}
\par For theoritical understanding of a tagger. We learned how a Hidden Markov Model tagger\cite{hmm}
 works. Here for this problem we used NLTK's default implementation of tagger(\textit{
 nltk.tag()}) as it is recommeneded for better results.

\begin{figure}[htp]
\centering
\includegraphics[height=2.5in,width=5in]{postag.png}
\caption{Parts-of-Speech Tagging}
\label{pos}
\end{figure}

\subsection{Named Entity Chunking\footnote{For more details: refer \cite{BookIE}}}
\par After tagging comes the named entity chunking. We'll group the pos tags into named
 entities (if possible intuit its class). Chunker is also a tagger that is trained on some
  corpus.
One of the most useful sources of information for NP-chunking(Noun Phrase-chunking) is part-of-speech tags.
This is one of the motivations for performing part-of-speech tagging in our information extraction system
\par Here for this problem we used NLTK's default implementation of named entity chunker
(\textit{nltk.ne\_chunk()}) as it is recommended for better results. It can be used for multi-
class(PERSON, LOCATION, ORGANIZATION and GPE) or binary class(NP).

\begin{figure}[htp]
\centering
\includegraphics[height=1in,width=5in]{chunk1.png}
\caption{Chunking}
\label{Chunk}
\end{figure}

\par For more insights on develping a chunker, please refer \cite{BookIE}. Here it described
 how to create a basic chunker and a chunker that can learn from data for good performance.
\par Morphology of words to identify Noun Phrases. And to identify the class of Noun Phrase(Named Entity) the chunker will use context of the Noun Phrase.
There are many formats to represent Named Entities, those are IB(Every token is \textbf{I}n the chunk or \textbf{B}egining of the chunk), IOB(Every token is \textbf{I}n the chunk or \textbf{O}ut of the chunk or \textbf{B}egining of the Chunk),tree representation etc.

\chapter{Relation Extraction}
\par Relation Extraction is an important component of Information Extraction.
\par Using the Named Entities and clever patterns we extract relation. These rules can 
get high precision as they are specific.
\par We will focus on the simpler task of extracting \textbf{relation triples}. Relation
triples are of the form \textit{(Named Entity,Relation,Named Entity)}.
\par We will use \textit{relextract} module of \textit{Python} for this.
\par Rule based relation extraction
We can create new structured knowledge bases by relation extraction
Questions that are asked in natural language can be converted in to a query to a structured
knowledge base.\\
\par So, Here is a question for us. \textit{Which relations should we extract ?} It depends
on how many classes of entities we are able to extract in \textit{Named Entity Recognition}.

But, a set of relations  comes from the \textit{Automated Content Extraction (ACE)} task.
\begin{figure}[htp]
\centering
\includegraphics[height=3in,width=5.2in]{ACE.jpg}
\caption{ACE Relation set}
\label{ACE}
\end{figure}
\ref{ACE}\footnote{Slide from: \url{https://class.coursera.org/nlp/lecture/138}}
\begin{figure}[htp]
\centering
\includegraphics[height=1in,width=3in]{rel.png}
\caption{Relation between PERSON and ORGANIZATION named entity classes}
\label{IE}
\end{figure}

Figure \ref{IE}\footnote{\url{https://class.coursera.org/nlp/lecture/139}} gives basic intuition of relation between entities.

\section{How to extract relations ?}
We can use:
\begin{itemize}
\item Hand-written patterns
\item Supervised,semi-supervised and unsupervised machine learning
\end{itemize}
We are using \textit{only} Hand-written patterns to extract relations as it is simplest
way. Here we are trying to extract relations between \textit{specific entities}.
\par We used \textbf{regular expressions } \cite{re} in Python to write patterns to extract relations.

\subsection{Procedure followed}

\begin{itemize}
\item First identify the named entities (POS tagging then Chunking).
\item Then we group a Noun phrase with its left context.\\ Now we'll have document as list of tuples.\\
Ex: \texttt{[(String1,Named Entity1), (String2,Named Entity2) , (String3,Named Entity3) , ... ]}\\
Here \texttt{Named Entity1,Named Entity2,Named Entity3} are tree representations of Noun Phrase.
Now we take two consecutive tuples and add them to create \textit{semi}relation dictionaries.\\
That dictionary contains \textbf{key},\textbf{value} as described in table \ref{table}
	\begin{table}[h]
	\centering
    \begin{tabular}{ | p{5cm} | p{5cm} |}
    \hline
    \textbf{KEY} & \textbf{VALUE} \\\hline
    \texttt{filler}(text between two Named Entities) &  \texttt{String2} which is leftcontext of Named Entity2\\ \hline
    \texttt{lcon}(left context of the relation) & \texttt{String1} which is leftcontext of Named Entity1.  \\\hline
    \texttt{objclass}(Class of object of the relation) & \texttt{Root of the Named Entity2 tree structure}\\\hline
    \texttt{objsym}(Normalized text of object with no white space & \texttt{Normalized object with underscore in palce of space.}\\\hline
    \texttt{objtext} & \texttt{objtext}\\\hline
    \texttt{rcon}(right context of the relation) & \texttt{String3}, which is right context of the Named Entity 2\\\hline
    \texttt{subjclass}(Class of subject of the relation) & \texttt{Root of the Named Entity1 tree structure.}\\\hline
    \texttt{subjsym}(Normalized text of suject with no white space) & \texttt{Normalized subject with underscore in palce of space.}\\\hline
    \texttt{subjtext} & \texttt{subject text}\\\hline
    \texttt{untagged\_filler} & \texttt{filler with no POS tags}\\\hline
    \end{tabular}
    \caption{Explanation of Key,Value pairs of \texttt{semi}relation dictionary}
    \label{table}
    \end{table}
\item Apply hand-written rules on filler,right context and left context to extract relations
\end{itemize}
\par For more information on relation extraction, please refer \cite{BookIE} for more information.
The rules we wrote for this project can be found here\footnote{\url{https://github.com/saikiran638/MyProjects/blob/master/FinalYearProject/RelationRules.py}}
\subsection{Positives of Hand-written rules}
\begin{itemize}
\item String patterns tends to be  high-precision.
\item Works well for specific domains/entities.
\end{itemize}

\subsection{Negatives of Handiwritten rules}
\begin{itemize}
\item It's difficult to think of all possible patterns.
\item We don't want to fix the entities for relation extraction.
\end{itemize}

\chapter{Results and Observations}
\section{Observations}
\par Named Entity Chunker provided in Python's NLTK (\textit{nltk.ne\_chunk}) considers morphology of 
words while chunking. Make sure that words are not normalized(HMMTagger requires words to 
be normalized) while chunking.
\par Trained HMMTagger(for PoS tagging) available in Python's NLTK with \textit{treebank} tagged corpus and
 tried chunking but performance is not as good as NLTK's Recommended PoS tagger (\textit{nltk.tag}).
 So, we used NLTK's Recommended PoS tagger for tagging sentences. It was found that NLTK's
  current recommended tagger is 'Averaged Perceptron Tagger'(It might change over time).
 
\section{Results}
\par This project is hosted on \texttt{github}\footnote{\url{https://github.com/saikiran638/MyProjects/tree/master/FinalYearProject}}
, you can access source code and documents of it.
\par For testing we used \textit{BBC}'s corpus.

\chapter{Conclusion and Future Work}
\section{Conclusion}
\par From the text we are going to extract information should a formal writing. For informal
 writings it won't work well as every steps assumes the formal nature of the text.

\section{Future Work}
\par Indian languages have very less \textit{tagged} corpus compared to English. We require
large amount of \textit{tagged corpus} to train classifiers.
\begin{itemize}
\item To apply Named Entity Recognition \& Information Extraction for Indian languages.
\end{itemize}

\begin{thebibliography}{99}
\bibitem{BookIE}
\textbf{Extracting Information from Text}\\
\url{http://www.nltk.org/book/ch07.html}

\bibitem{tag}
\textbf{Categorizing and Tagging Words}\\
\url{http://www.nltk.org/book/ch05.html}

\bibitem{howtoRE}
\textbf{Information Extraction in NLTK}\\
\url{http://www.nltk.org/howto/relextract.html}

\bibitem{Videos}
Prof.Yaser Abu-Mostafa,California Institute of Technology\\
\textbf{Learning from Data}(Online Course)

\bibitem{re}
\textbf{Regular Expression in Python}\\
\url{https://docs.python.org/2/howto/regex.html#regex-howto}

\bibitem{hmm}
Prof. Michael Collins,Columbia University\\
\textbf{Tagging Problems, and Hidden Markov Models}\\
\url{http://www.cs.columbia.edu/~mcollins/hmms-spring2013.pdf}

\bibitem{mc}
Prof. Michael Collins, Columbia University\\
\textbf{Natural Language Processing}(coursera)\\
\url{https://class.coursera.org/nlangp-001/lecture}

\bibitem{SNLP}
\textbf{Dan Jurafsky,Christopher Manning}\\
Week 4 - Relation Extraction of \textbf{Natural Language Processing}(coursera lectures),\\
\url{https://class.coursera.org/nlp/lecture}
\bibitem{Rahul}
Rahul Sharnagat\\
\textbf{Named Entity Recognition: A Literature Survery}\\
\url{http://www.cfilt.iitb.ac.in/resources/surveys/rahul-ner-survey.pdf}
\end{thebibliography}
\end{document}
% bTEzX3BoeXNpY3M6cGFyaW1p
