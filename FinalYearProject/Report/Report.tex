	
\documentclass[12pt]{report}

\usepackage{amsmath}
\usepackage[colorlinks,linkcolor=black]{hyperref}
\usepackage[margin=2cm]{geometry}
%for images
\usepackage{graphicx}
%\usepackage{wrapfig}
\usepackage{url}

\newcommand*{\Signature}[1]{
	\vspace{3cm}
    \par\noindent\makebox[5cm]{\hrulefill}
    \par\noindent\makebox[5cm][l]{#1}
}


\newcommand*{\plogo}{\fbox{$\mathcal{PL}$}} % Generic publisher logo

%----------------------------------------------------------------------------------------
%	TITLE PAGE
%----------------------------------------------------------------------------------------

\newcommand*{\titleGP}{\begingroup % Create the command for including the title page in the document
\centering % Center all text
\vspace*{\baselineskip} % White space at the top of the page


\rule{\textwidth}{1.6pt}\vspace*{-\baselineskip}\vspace*{2pt} % Thick horizontal line
\rule{\textwidth}{0.4pt}\\[\baselineskip] % Thin horizontal line

{\LARGE \textbf{Entity Recognition\\ and \\[0.3\baselineskip] Relation Extraction}}\\[0.2\baselineskip] % Title

\rule{\textwidth}{0.4pt}\vspace*{-\baselineskip}\vspace{3.2pt} % Thin horizontal line
\rule{\textwidth}{1.6pt}\\[\baselineskip] % Thick horizontal line

%\scshape % Small caps
{\Large Bachelor's Thesis\par} % Tagline(s) or further description
 submitted in partial fullfilment of\\
 the requirements for the degree of \\
Bachelor of Technology\\
in\\
\textbf {Computer Science and Engineering}\\
[\baselineskip]
By\\
{\Large SAI KIRAN NARAGAM\\N100638\par}
\\[\baselineskip] % Tagline(s) or further description

{Under the Guidance of\par}
{\textbf {Mr. A.Udaya Kumar\\Asst. Professor\\Dept. of Computer Science and Engineering\\  RGUKT Nuzvid}\par}
\vspace*{2\baselineskip} % Whitespace between location/year and editors


%Edited by \\[\baselineskip]
%{\Large JOHN SMITH \\ JANE SMITH \\ JAMES SMITH\par} % Editor list
%{\itshape The University of California \\ Berkeley\par} % Editor affiliation

\vfill % Whitespace between editor names and publisher logo
\includegraphics[scale=.4]{rguktlogo.jpg}\\
%\plogo \\[0.3\baselineskip] % Publisher logo
{\scshape \textbf{Rajiv Gandhi University of Knowledge Technologies, Nuzvid.\\Krishna District,Andhra Pradesh.}} %\\[0.3\baselineskip] % Year published
%{\large THE PUBLISHER}\par % Publisher

\endgroup}

%\title{Scilab Optimization Toolbox Using Symphony\\Submitted in Fulfilment of\\Summer Internship 2015 Project}
%\author{Keyur Joshi \and Sai Kiran Naragam \and Iswarya Hariharan}
%\date{May-July 2015}

\usepackage{fancyhdr, graphicx}
\renewcommand{\headrulewidth}{0pt}
\fancyhead[R]{\textbf{Rajiv Gandhi University of Knowledge Technologies\\
	(A.P. Government Act 18 of 2008)\\
	RGUKT NUZVID, Krishna District - 521202}\\}
\fancyhead[L]{
\includegraphics[width=1.5cm]{rguktlogo.jpg}
}

\begin{document}

\pagestyle{empty} % Removes page numbers

\titleGP % This command includes the title page

% title page
%\maketitle
\cleardoublepage
\pagenumbering{roman}
\setcounter{page}{1} %reset the page counter
%acknowledgements

%\chapter*{Project Approval}
\newgeometry{left=3cm,right=3cm,top=5cm}


\section*{}
%\begin{center}\textbf{Rajiv Gandhi University of Knowledge Technologies\\(A.P. Government Act 18 of 2008)\\RGUKT-NUZVID, Krishna Dist. - 521202\\}\end{center}

\thispagestyle{fancy}
\begin{center}\makebox[15cm]{\hrulefill}\end{center}
\vspace{1cm}
\begin{center}
\subsection*{CERTIFICATE OF COMPLETION\\\makebox[10cm]{\hrulefill}}

\end{center}
\par This is to certify that the work entitled, \textbf{ Entity  Recognition and Relation Extraction} 
is the bona fied work of \textbf{NARAGAM SAI KIRAN}, ID No: \textbf{N100638}, carried out under my guidance and supervision,
for the partial fulfillment of the requirements for the award of the degree of Bachelor of Technology in Computer Science and Engineering. 

\Signature{Mr. Ambati Udaya Kumar,}\\Assistant Professor,\\Dept.of CSE.

\pagebreak
\newgeometry{left=3cm,right=3cm,top=5cm}
\section*{}
\thispagestyle{fancy}
\begin{center}\makebox[15cm]{\hrulefill}\end{center}
\vspace{1cm}
\begin{center}
\subsection*{CERTIFICATE OF EXAMINATION\\\makebox[10cm]{\hrulefill}}

\end{center}
\par This is to certify that the work entitled, \textbf{"Entity Recognition and Relation Extraction"}
is the bonafide work of \textbf{NARAGAM SAI KIRAN}, ID No: \textbf{N100638} and here by accord our approval
 of it as a study carried out and presented in a manner required for its
  acceptance in the partial fulfillment of the requirement for 
  the award of the degree of Bachelor of Technology for which it has been submitted.
   This approval does not necessarily endorse or accept every statement made, opinion expressed or conclusion drawn, as a recorded in this thesis.
    It only signifies the acceptance of this thesis for the purpose for which it has been submitted.

\Signature{Mr. Ambati Udaya Kumar,}\\Assistant Professor,\\Dept.of CSE.
\Signature{Mr. Amit Patel,}\\Lecturer,\\Dept.of CSE.
\pagebreak

\newgeometry{left=3cm,right=3cm,top=5cm}
\section*{}
\thispagestyle{fancy}
\begin{center}\makebox[15cm]{\hrulefill}\end{center}
\vspace{1cm}
\begin{center}
\subsection*{DECLARATION\\\makebox[5cm]{\hrulefill}}

\end{center}
\par I \textbf{NARAGAM SAI KIRAN}, with ID No:\textbf{N100638} hereby declare that t
he project report entitle \textbf{Entity Recognition and Relation Extraction} done by me
 under the guidance of \textbf{Mr. Ambati Udaya Kumar,M.Tech} is submitted for the partial
  fulfillment of the requirements for the award of the degree of Bachelor of Technology
   in Computer Science and Engineering during the academic session August 2015 – April 2016 at RGUKT – Nuzvid.

\par I also declare that this project is a result of my own effort and has not been copied or imitated from any source.
 Citations from any websites are mentioned in the references.
\par The results embodied in this project report have not been submitted to any other university or institute for the award of any degree or diploma.

\Signature{N. SAI KIRAN,}\\N100638.\\
\begin{flushright}
Date : \makebox[3cm]{\hrulefill}\\
Palce: \makebox[3cm]{\hrulefill}\\
\end{flushright}
\pagebreak

\newgeometry{margin=2cm}
\chapter*{Acknowledgements}
\par I would like to thank my parents,guide and friends
\chapter*{Abstract}

\par Abstract goes here.

%table of contents
\pagestyle{empty} %get rid of header/footer for toc page
\tableofcontents %put toc in
\cleardoublepage %start new page
\pagestyle{plain} % put headers/footers back on
\pagenumbering{arabic}
\setcounter{page}{1} %reset the page counter

\chapter{Introduction}
When we have large amount of (previous)data we might want to extract some useful information
 out of it, and use it as summary. or we can predict the future events by learning from the
 data at hand.\\
Most of the time data that is available for use in un-structured form like Natural Language Text
rather than structured form like tables. It is easy to extract required information or 
answer a question if the data we are working on has structured form.

But it is difficult to handle unstructured data. Because Natural Language Processing(NLP)
that works on unstructured data is still developing.

The amount of natural language text that is available in electronic 
form is truly staggering, and is increasing every day. 
However, 
the complexity of natural language can make it very difficult to access the information in that text\cite{BookIE}.\\

\par If we instead focus our efforts on a limited set of questions or 
"entity relations," such as "where are different facilities located,
" or "who is employed by what company," we can make significant progress.\cite{BookIE}


\begin{figure}[htp]
\centering
\includegraphics[height=2in,width=3in]{ie.png}
\caption{Simple Pipeline Architecture for an Information Extraction System\cite{BookIE}}
\label{IE}
\end{figure}

\chapter{Entity Recognition}
\section{Entity Recognition as Tagging}
\par Bikel et. al
\begin{figure}[htp]
\centering
\includegraphics[scale=.7]{nerastagging2.png}
\caption{NER as Tagging}
\label{NERasT}
\end{figure}

\section{Entity Recognition with chunking}
\par Here goes ER
\subsection{PoS Tagging}
POS
\subsection{Named Entity Chunking}
\begin{figure}[]
\centering
\includegraphics[scale=.2]{chunk1.png}
\caption{Chunking}
\label{Chunk}
\end{figure}
\chapter{Relation Extraction}
\par Relation Extraction is an important component of Information Extraction.
\par Using the Named Entities and clever patterns we extract relation. These rules can 
get high precision as they are specific.
\par We will focus on the simpler task of extracting \textbf{relation triples}. Relation
triples are of the form \textit{(Named Entity,Relation,Named Entity)}.
\par We will use \textit{relextract} module of \textit{Python} for this.
\par Rule based relation extraction
We can create new structured knowledge bases by relation extraction
Questions that are asked in natural language can be converted in to a query to a structured
knowledge base.\\
\par So, Here is a question for us. \textit{Which relations should we extract ?} It depends
on how many classes of entities we are able to extract in \textit{Entity Recognition}.

But, a set of relations  comes from the \textit{Automated Content Extraction (ACE)} task.
\begin{figure}[htp]
\centering
\includegraphics[height=3in,width=5.2in]{ACE.jpg}
\caption{ACE Relation set}
\label{ACE}
\end{figure}
\ref{ACE}\footnote{Slide from: \url{https://class.coursera.org/nlp/lecture/138}}
\begin{figure}[htp]
\centering
\includegraphics[height=1in,width=3in]{rel.png}
\caption{IE system}
\label{IE}
\end{figure}

IE System \ref{IE}\footnote{https://class.coursera.org/nlp/lecture/139}

\section{How to extract relations ?}
We can use:
\begin{itemize}
\item Hand-written patterns
\item Supervised,semi-supervised and unsupervised machine learning
\end{itemize}
We are using \textit{only} Hand-written patterns to extract relations as it is simplest
way. Here we are trying to extract relations between \textit{specific entities}. 
\subsection{Positives of Hand-written rules}
\begin{itemize}
\item String patterns tends to be  high-precision.
\item Works well for specific domains/entities.
\end{itemize}

\subsection{Negatives of Handiwritten rules}
\begin{itemize}
\item It's difficult to think of all possible patterns.
\item We don't want to fix the entities for relation extraction.
\end{itemize}

\chapter{Results and Observations}
\chapter{Conclusion and Future Work}


\begin{thebibliography}{99}
\bibitem{BookIE}
\textbf{Extracting Information from Text}\\
\url{http://www.nltk.org/book/ch07.html}

\bibitem{howtoRE}
\textbf{Information Extraction in NLTK}\\
\url{http://www.nltk.org/howto/relextract.html}

\bibitem{Videos}
Prof.Yaser Abu-Mostafa,California Institute of Technology\\
\textbf{Learning from Data}(Online Course)

\bibitem{re}
\textbf{Regular Expression in Python}\\
\url{https://docs.python.org/2/howto/regex.html#regex-howto}

\bibitem{h}
Prof. Michael Collins,Columbia University\\
\textbf{Tagging Problems, and Hidden Markov Models}\\
\url{http://www.cs.columbia.edu/~mcollins/hmms-spring2013.pdf}

\bibitem{SNLP}
\textbf{Dan Jurafsky,Christopher Manning}\\
Week 4 - Relation Extraction of \textbf{Natural Language Processing}(coursera lectures),\\
\url{https://class.coursera.org/nlp/lecture}
\end{thebibliography}
\end{document}
% bTEzX3BoeXNpY3M6cGFyaW1p
